\documentclass{cheatsheet}

\begin{document}
	\begin{gather*}
		\left|\frac{\partial(x,y,z)}{\partial(r,\theta,\phi)}\right| = r^2 \sin \theta \\
		\left| \frac{\partial(x,y)}{\partial(r,\theta)} \right| = r
	\end{gather*}

	\section{Conservation of Momentum}
	\begin{gather}
		\vec p = m\vec v	\tag{momentum}\\
		\dot{\vec P} = \vec F_\mathrm{ext} \tag{3rd law}
	\end{gather}
	\subsection{Collisions}
		An \emph{elastic} collision conserves kinetic energy.
		\[
			m_1\vec v_1 + m_2\vec v_2 = m_1\vec v_1\,' + m_2\vec v_2\,'
		\]
		An \emph{inelastic} collision does not.
		\[
			m_1\vec v_1 + m_2\vec v_2 = m\vec v\,'
		\]
	\section{Changing Mass idk}
	\section{Center of Mass}
	\begin{gather*}
		M = \int \rho \mathrm{d}V	\tag{total mass}\\
		\vec R =  \frac{1}{M_\mathrm{tot}}  \int \vec r \mathrm{d}m \tag{location of CoM}\raisetag{-6pt}\\
	\intertext{$\mathrm{d}m$ can usually be rewritten in terms of $\mathrm{d}V$}\\
		\vec R = \frac{1}{M_\mathrm{tot}} \int \vec r \rho(\vec r) \mathrm{d}V
	\end{gather*}
	Look for symmetries by which you can substitute $\vec R$ and $\vec r$ for scalars.

\end{document}