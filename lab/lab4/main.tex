\documentclass{article}


\title{Resistors, Inductors, and Capacitors in Circuits}
\author{Raymond Langehennig}

\begin{document}
\maketitle

	\section{Introduction}
		\subsection{Background}
			In DC circuits, resistance is the only form of opposition the current faces.\\
			In AC circuits, however, the oscillating nature of the current induces a second form of opposition called \emph{reactance}. This occurs mostly in inductors \& capacitors, though the exact mechanism by which the alternating current causes this is different in the two components.
			
			\subsubsection*{Inductors}
				Inductors are components used for a variety of reasons (energy storage, magnetic field manipulation, filters), but at their simplest they consist of a conductive material coiled around a "core" of some material that serves to confine the magnetic field induced by the current running through the conductor better.\\
				Current gives rise to a magnetic field as illustrated by Amp\`ere's Law. It stands to reason, then, that a \emph{changing} current (such as AC) gives rise to a \emph{changing} magnetic field. A changing magnetic field, however, is opposed due to Lenz's Law, resulting finally in a "back-emf" that diminishes the original current through the inductor. This is the \emph{inductive reactance}, symbolized by $X_C$.

			\subsubsection*{Capacitors}
				Capacitors, like inductors, also store energy, but they store it in the form of electric potential energy. This is generally accomplished through the separation of two conductors by a dielectric, between which an electric field arises when an electromotive force is supplied due to the accumulation of opposing charge on either conductor (electrons leaving one end/gathering on the other).
				The voltage across the two conductors approaches that of the supplied electromotive force, while the current in the circuit approaches 0. 
				In a DC circuit,


		\subsection{Objective}
			The object of this lab is to gain an understanding of and familiarity with and reactance in resistor-capacitor (RC), resistor-inductor (RL), and resistor-inductor-capacitor (RLC) circuits. This will be achieved mainly through the use of a breadboard, function generator, and oscilloscope.

		\subsection{Theory}
			dwadawd

	\section{Procedure}
		

	\section{Data}
		\subsection{RL}
		
	\section{}

	\section{Questions}

\end{document}