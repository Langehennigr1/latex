\documentclass{article}
\usepackage{booktabs}
\usepackage{textgreek}

\title{Resistors, Inductors, and Capacitors in Circuits}
\author{Raymond Langehennig}

\begin{document}
\maketitle

	\section{Introduction}
		\subsection{Background}
			In DC circuits, resistance is the only form of opposition the current faces.\\
			In AC circuits, however, the oscillating nature of the current induces a second form of opposition called \emph{reactance}. This occurs mostly in inductors \& capacitors, though the exact mechanism by which the alternating current causes this is different in the two components.
			
			\subsubsection*{Inductors}
				Inductors are components used for a variety of reasons (energy storage, magnetic field manipulation, filters), but at their simplest they consist of a conductive material coiled around a "core" of some material that serves to confine the magnetic field induced by the current running through the conductor better.\\
				Current gives rise to a magnetic field as illustrated by Amp\`ere's Law. It stands to reason, then, that a \emph{changing} current (such as AC) gives rise to a \emph{changing} magnetic field. A changing magnetic field, however, is opposed due to Lenz's Law, resulting finally in a "back-emf" that diminishes the original current through the inductor. This is the \emph{inductive reactance}, symbolized by X\textsubscript{L}.

			\subsubsection*{Capacitors}
				Capacitors, like inductors, also store energy, but they store it in the form of electric potential energy. This is generally accomplished through the separation of two conductors by a dielectric, between which an electric field arises when an electromotive force is supplied due to the accumulation of opposing charge on either conductor (electrons leaving one end/gathering on the other).
				In a DC circuit, the voltage across the two conductors approaches that of the supplied electromotive force, while the current in the circuit approaches 0 (as there is no potential difference to cause a flow of charge).
				In an AC circuit, charge will build to an extent before the polarity of the voltage source switches and the same charge builds on the other terminal of the capacitor. This means that a specific amount of charge will accumulate each cycle (in proportion to the length of that cycle) allowing the capacitor \& emf source to reach a specific difference in potential. This potential difference will be necessarily \emph{less} than that between the emf source \& capacitor when it is entirely discharged, meaning that the current will be less in proportion to the charge in the capacitor per cycle and therefore 


		\subsection{Objective}
			The object of this lab is to gain an understanding of and familiarity with and reactance in resistor-capacitor (RC), resistor-inductor (RL), and resistor-inductor-capacitor (RLC) circuits. This will be achieved mainly through the use of a breadboard, function generator, and oscilloscope.

		\subsection{Theory}
			dwadawd

	\section{Procedure}
		For all experiments, a Global Specialties PB-505 breadboard was used in conjunction with a RIGOL DS1104Z oscilloscope \& DG1022Z function generator. A multitude of electrical components were also used to construct the circuits, the specifications of which will be laid out below.
		\begin{table}[htbp]
	\centering
	\caption{Components}
	\begin{tabular}{ccc}
		\hline
		Component & Nominal value & Actual value \\ \hline
		Resistors & 220 \textOmega & $218.9  \pm awawawaw\Omega$ \\
		multirow & 1 k\textOmega & $1 \pm awawa\mathrm{k\Omega}$ \\
		Capacitor & $0.1 \mathrm{\mu F}$ & $0.1023 \pm \mathrm{\mu F}$ \\
		Inductor & A & A \\ \hline
	\end{tabular}
\end{table}

		\subsection{RC Circuit}
			An RC circuit was constructed with the $220 \Omega$ resistor \& $0.1 \mu$ capacitor in series. The probe of the oscilloscope was placed across the resistor and a 5 V sinusoidal signal was generated. The peak voltage across the resistor was measured at 100, 200, and 400 Hz \&


	\section{Data}
		\subsection{RL}
			\begin{table}[htbp]
	\centering
	\caption{Phase Shift}
	\begin{tabular}{ccccc}
		\hline
		A & A & A & A & A \\ \hline
		A & A & A & A & A \\
		A & A & A & A & A \\
		A & A & A & A & A \\ \hline
	\end{tabular}
\end{table}

		
	\section{}

	\section{Questions}
		\begin{quotation}
			The peak current in your circuit is not constant with frequency. Why is this based on the physics of a capacitor?
		\end{quotation}
		

	\section{Results}
	\section{Conclusion}
	\section*{Tables}

		\begin{table}[]
			\begin{tabular}{@{}rrrrr@{}}
			\toprule
			\multicolumn{1}{l}{f} & \multicolumn{1}{l}{X\textsubscript{C} (k\textOmega)} & \multicolumn{1}{l}{I\textsubscript{pp} (mA)} & \multicolumn{1}{l}{V\textsubscript{pRmeas} (mV)} & \multicolumn{1}{l}{I\textsubscript{ppR,meas} (mA)} \\ \midrule
			100 & 15.56 & 0.32 & 34.80  & 0.16 \\
			200 & 7.78  & 0.64 & 69.20  & 0.32 \\
			400 & 3.89  & 1.29 & 134.00 & 0.61 \\ \bottomrule
		\end{tabular}
	\end{table}
\end{document}