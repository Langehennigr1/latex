\documentclass{article}
\usepackage{booktabs}
\usepackage{todonotes}
\usepackage{graphicx} % Required for inserting images
\usepackage[letterpaper,top=2cm,bottom=2cm,left=3cm,right=3cm,marginparwidth=1.75cm]{geometry}
\usepackage{hyperref}
\usepackage{subcaption}
\usepackage{enumitem}
\usepackage{float}

\title{Introduction to Oscilloscopes \& Function Generators}
\author{Raymond Langehennig}

\begin{document}

\maketitle

\section{Introduction}
    \subsection{Background}
        The measurement of electric phenomena has progressed from rudimentary galvanometers or even compasses, the deflection of which being the measurement taken, to complicated digital instruments such as the \emph{oscilloscope}, which is the subject of this lab.\\
        Oscilloscopes are devices which enable the waveform of an electric signal passed through them to be inspected, usually across very small intervals of time. Prior to their invention, this could only be achieved by manually recording readings from a galvanometer at different points of time. However, with the discovery of cathode rays (electron beams) and the effect of magnets on their path, the first analog oscilloscope was invented by passing these electron beams between two pairs of conductive plates and onto a fluorescent screen at the end of the vacuum tube. These pairs of plates, one vertical \& the other horizontal, could effectively control the deflection of the beam and therefore the spot that fluoresced on the screen through the application of current. One could therefore connect the vertical plates to the signal to be measured and the horizontal plates to another signal (usually a sawtooth wave) so that the trace would rapidly sweep across the screen, resulting in a wave shape. Digital oscilloscopes would follow later, by sampling the voltage reading many times per second and plotting these over time.

    \subsection{Objective}
    The aim of this lab is to gain experience with digital oscilloscopes and to better understand how they can be used to take measurements. In particular, the oscilloscope will be used to measure the time constant of a capacitor.
    

\section{Procedure}
    This section details both the inital experimentation with the oscilloscope in order to become familiar with it and the subsequent experiment with the charging and discharging of a capacitor.\\ % rewrite!!
    For both experiments, a RIGOL DS1104-Z oscilloscope and DG1022-Z function generator (pictured in figure \ref{oscnfunc1}) were used.

    \begin{figure}[h]
        \begin{minipage}{\textwidth/2}
            \includegraphics[width=\linewidth]{WIN_20240927_13_48_12_Pro.jpg}
            \caption{The oscilloscope (top) connected to the function generator (bottom) via coaxial cable.}
            \label{oscnfunc}
        \end{minipage}
        \begin{minipage}{\textwidth/2}
            \includegraphics[width=\linewidth]{WIN_20240927_13_51_59_Pro.jpg}
            \caption{The interface of the function generator.}
            \label{funcgen}
        \end{minipage}
    \end{figure}

    \subsection{Oscilloscope basics}
        In order to generate a signal to view in the oscilloscope, the function generator was configured as shown in figure \ref{funcgen} to produce a 2.5 V, 123 Hz sine wave.

        \subsubsection{Exercises}
            A series of exercises were carried out to gain familiarity with the oscilloscope:

            \begin{enumerate}[label=(\alph*)]
                \item The function generator was set to produce an 80 Hz sine wave (fig. \ref{exa}).
                \begin{figure}[p]
                    \begin{minipage}{\textwidth/2}
                        \includegraphics[width=\linewidth]{a.jpg}
                        \caption{Exercise A}
                        \label{exa}
                    \end{minipage}
                    \begin{minipage}{\textwidth/2}
                        \includegraphics[width=\linewidth]{b.jpg}
                        \caption{Exercise B}
                        \label{exb}
                    \end{minipage}
                \end{figure}

                \item The amplitude was changed (fig. \ref{exb}).
                
                \item The frequency was increased to 800 Hz (fig. \ref{exc}).
                \begin{figure}[p]
                    \centering
                    \includegraphics[width=\textwidth/2]{c.jpg}
                    \caption{Exercise C}
                    \label{exc}
                \end{figure}
                
                \item A DC offset was added (fig. \ref{exd}).
                \begin{figure}[p]
                    \begin{subfigure}{\textwidth/2}
                        \includegraphics[width=\linewidth]{d v.jpg}
                        \caption{}
                    \end{subfigure}
                    \begin{subfigure}{\textwidth/2}
                        \includegraphics[width=\linewidth]{d offset.jpg}
                        \caption{}
                    \end{subfigure}
                    \caption{Exercise D}
                    \label{exd}
                \end{figure}
                
                \item A 80 Hz square wave was produced (fig. \ref{exe})
                \begin{figure}[p]
                    \begin{minipage}{\textwidth/2}
                        \includegraphics[width=\linewidth]{e.jpg}
                        \caption{Exercise E}
                        \label{exe}
                    \end{minipage}
                    \begin{minipage}{\textwidth/2}
                        \includegraphics[width=\linewidth]{f.jpg}
                        \caption{Exercise F}
                        \label{exf}
                    \end{minipage}
                \end{figure}

                \item The frequency was increased to 8 kHz (fig. \ref{exf}).

                \item A 200 Hz sine wave was produced and fed into both CH1 \& CH2 of the oscilloscope (fig. \ref{exg}).
                \begin{figure}[p]
                    \begin{subfigure}{\textwidth/2}
                        \includegraphics[width=\linewidth]{g ch1.jpg}
                        \caption{CH1}
                    \end{subfigure}
                    \begin{subfigure}{\textwidth/2}
                        \includegraphics[width=\linewidth]{g ch2.jpg}
                        \caption{CH2}
                    \end{subfigure}
                    \caption{Exercise G}
                    \label{exg}
                \end{figure}
            \end{enumerate}




    \subsection{Charging \& discharging of a capacitor}
        Thus far, only the output of the function generator had been investigated using the oscilloscope. In actual practice, oscilloscopes are used in a similar manner to a voltmeter---that is, they are connected in parallel to the component across which the voltage is to be measured.\\ % grammar?
        For this portion of the experiment, a circuit would be constructed on a breadboard consisting of two parallel LEDs, a 997 $\Omega$ resistor, and 100.9 nF capacitor in series with the output of the function generator as shown in figure \ref{circuitry}.
        The function generator was configured to produce an 8 V, 500 Hz square wave which was output to both the circuit \& the first channel of the oscilloscope by means of a Tee-splitter.
        Across the capacitor was also placed a probe, which was connected to the second channel of the oscilloscope.\\

        

        %kinda fucked
        \begin{figure}[H]
            \begin{subfigure}{\textwidth/3}
                \includegraphics[width=\linewidth]{g115.png}
                \caption{}
            \end{subfigure}
            \begin{subfigure}{\textwidth/3}
                \includegraphics[width=\linewidth]{breadboard.png}
                \caption{}
            \end{subfigure}
            \begin{subfigure}{\textwidth/3}
                \includegraphics[width=\linewidth]{circuit.jpg}
                \caption{}
            \end{subfigure}
            \caption{The circuit used to charge \& discharge the capacitor (a) as an ideal circuit, (b) sketched, and (c) assembled.}
            \label{circuitry}
        \end{figure}


\section{Results}
    The time constant $\tau$ of a capacitor is the time it takes for the capacitor to charge to $V_\mathrm{max}(1-e^-1)$ or discharge to $V_\mathrm{max}e^{-1}$. From this, it follows that if $V_\mathrm{max}$ is known, one can calculate $V(\tau)$ charging \& discharging and measure the time it takes to reach that voltage.
    For the constructed circuit, this was 2.56 V charging and 1.49 V discharging.
    In order to measure the time it takes to reach these voltages with the oscilloscope, the horizontal cursors were used to measure voltage relative to the minimum of the signal and then the vertical cursors were used to measure the time from the start of the charging/discharging cycle to the point where the horizontal cursors intersect the signal, as can be seen in figure \ref{measuringtau}\\

        \begin{figure}[h]
            \begin{subfigure}{0.5\linewidth}
                \includegraphics[width=\linewidth]{tau charging.jpg}
                \caption{}
            \end{subfigure}
            \begin{subfigure}{0.5\linewidth}
                \includegraphics[width=\linewidth]{tau discharging.jpg}
                \caption{}
            \end{subfigure}
            \caption{Measurement of $\tau$ for (a) charging \& (b) discharging.}
            \label{measuringtau}
        \end{figure}

    Through this method, two figures for $\tau$ were obtained, 116.0 $\mu$s charging \& 120.0 $\mu$s, which differ from each other by 3\%.


\section{Conclusion}
    The theoretical time constant can be calculated by multiplying the resistance \& capacitance of the circuit to get 101 $\mu$s. This gives us a percent error of 16\%.\\
    
    The source of this error was not initally apparent. Deterioration of the components, for example, can be ruled out because the nominal resistance \& capacitance (1 k$\Omega$ \& 100 nF) gives a theoretical $\tau$ of 100 $\mu$s, still leaving at best a 16 $\mu$s discrepancy. This could perhaps be attributed to the other components in the circuit, particularly the LEDs as they do not follow Ohm's law.

\end{document}